%-------------------------------------------------------------------------------
%	INTRO
%-------------------------------------------------------------------------------

\begin{intro}
\addchaptertocentry{\introname}
\bigskip

Өнөө үед хүн бүрт нэг газраас нөгөө газарт ямар нэгэн ачаа бараа болон шуудан зэргийг хүргэх хэрэгцээ шаардлага гардаг билээ. Гэвч тэр болгон хүргэлтийн үйлчилгээ үзүүлдэг байгууллага байхгүй, жижиг ачаа барааг хүргэдэггүй мөн удаан хугацаагаар хүлээдэг гэх мэт дутмаг талуудаас болж бүрэн үйлчилгээ авч чаддаггүй.

%-----------------------------------
%	SUBSECTION 1
%-----------------------------------
\subsection*{Зорилго}
Уг төслийн хүрээнд дээрх асуудлуудыг орчин үеийн технологиудыг ашиглан шийдсэн хүргэлтийн үйлчилгээ үйлчилгээг бий болгох зорьж байна.

%-----------------------------------
%	SUBSECTION 2
%-----------------------------------
\subsection*{Зорилт}
Дээрх зорилгоо хэрэгжүүлэхийн тулд дараах зорилтуудыг дэвшүүлэн тавьж байна.
Үүнд:
\begin{itemize}[label={--}]
    \item Ижил төстэй системүүдийг судлах;
	\item Төслийг хөгжүүлэх технологийг судлан сонгох;
	\item Төслийг шаардлагыг тодорхойлох;
	\item Төслийг шинжилгээ ба зохиомжийг гаргах;
	\item Төслийг хөгжүүлэх;
\end{itemize}

%-----------------------------------
%	SUBSECTION 3
%-----------------------------------
\subsection*{Систем хөгжүүлэх үндэслэл}
Монгол улсын нийслэл Улаанбаатар хотын хүн амын тоо 1.4 саяд хүрч дэлхийн их хотуудын тоонд зүй ёсоор багтаж байгаа билээ. Хүн амын тоогоо дагаад хотын түгжрэл болон хэт нягтаршилт нэмэгдсээр байна. Систем хөгжүүлэх үндэслэлийг хүргэгч болон үйлчлүүлэгч талаас  тодотгон харуулав:
\begin{itemize}
\item \textbf{Хүргэгч талаас:}
\begin{itemize}[label={--}]
\item\textit{Үйлчлүүлэгчдийн байршлыг мэддэггүй.}\\
Хүргэгчид үйлчлүүлэгчээ хаана байгааг баттай мэдэхгүй зөн совингоороо үйлчлүүлэгчээ удаан хугацаанд хайж олдгоос зардал нь нэмэгдэхэд хүргэдэг.
\end{itemize}
\item \textbf{Үйлчлүүлэгчийн талаас:}
\begin{itemize}[label={--}]
\item\textit{Үйлчилгээг түргэн хугацаанд авч чаддаггүй.}
\item\textit{Хүргэгчийг хаана явж байгааг хянах боломжгүй.}
\item\textit{Хүргэгч үйлчлүүлэгчийн хүрэх газрын байршлыг мэддэггүй.}
\end{itemize}
Энэхүү бэрхшээлүүд дээр үндэслэн хүргэлтийн үйлчилгээ эрхлэгч хүргэгч болон үйлчлүүлэгч нарын цаг хугацаа, зардлыг хэмнэх ухаалаг систем хөгжүүлэх үндэслэлийг гарган тавьж буй юм.
\end{itemize}

%-----------------------------------
%	SUBSECTION 4
%-----------------------------------
\subsection*{Судлагдсан байдал}
Одоогоор ижил төстэй апп монгол улсад үйл ажиллагаа явуулдаггүй ч гэсэн ижил төстэй системүүдийг гүүглэ плэйстоор болон аппстоороос олох боломжтой. Энэ нь энэхүү төрлийн аппликэйшнүүд хэр судлагдсаныг харуулж байна.

%-----------------------------------
%	SUBSECTION 5
%-----------------------------------
\subsection*{Шинэлэг тал}
Дээрх системүүдээс ялгарах шинэлэг талууд нь
\begin{itemize}[label={--}]
    \itemЯмар нэг үйлчилгээний компанид зориулагдаагүй.
    \itemХэрэглэхэд хялбар.
    \itemУхаалаг утастай хүн бүр ашиглах боломжтой.
    \itemҮйлчлүүлэгч нь хүргэлт хийх боломжтой.
    \itemХэрэглэгч бүр харилцан үнэлгээ өгөх боломжтой.
    \itemҮйлчлүүлэгч өөрийн хүргэгчийн байршилыг бодитоор хянах боломжтой.
\end{itemize}


%-----------------------------------
%	SUBSECTION 6
%-----------------------------------
\subsection*{Системийн ач холбогдол}
Уг систем нь практикт нэвтэрснээр дараах ач холбогдолтой. Үүнд: 
\begin{itemize}[label={--}]
    \itemМонгол улс дахь ухаалаг утсаа ашиглан үйлчилгээ авч буй иргэдийн тоо мэдэгдэхүйц нэмэгдэнэ.
    \itemМонгол улсын иргэдийн мэдээллийн технологийн хэрэглээ түүний хөгжилд багахан боловч хувь нэмрээ оруулж, системээр дамжуулан үйлчилгээ авч байгаа хүн бүрд баталгаатай, чанартай үйлчилгээг хүргэнэ.
    \itemДугуй унадаг хүмүүсийн тоо нэмэгдэнэ.
\end{itemize}


%-----------------------------------
%	SUBSECTION 7
%-----------------------------------
\subsection*{Системийн хамрах хүрээ}
Энэхүү систем нь 13 наснаас дээш, ухаалаг утастай хэн бүхэн манай системийг ашиглах бүрэн боломжтой юм.

\end{intro}

\clearpage